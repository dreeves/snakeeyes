\documentclass[12pt, letterpaper]{article}
\usepackage{amsmath}
\usepackage{hyperref}
\usepackage{amssymb}
\usepackage[table]{xcolor}% http://clightgray.org/pkg/xcolor
\hypersetup{
	colorlinks=true,
	linkcolor=blue,
	filecolor=magenta,
	urlcolor=cyan,
	}
\urlstyle{same}
\newcommand{\mycomment}[1]{}
\begin{document}
\mycomment{
uniform selection, with out replacement from 1,000,000 people. flip a coin.
\begin{center}\begin{tabular}{ r | c c | }
$\text{ }$ & heads $\frac{1}{2}$ & tails $\frac{1}{2}$  \\
\hline
$1E-6$ & $5E-7$ & $5E-7$ \\
$\text{Round 1 (one player)}$ & player dead & player won  \\
$\text{ }$ & $\text{ }$ & $\text{ }$  \\
$999999E-6$ & $4999995E-7$ &  $4999995E-7$\\
$\text{Round 2 (999999 players)}$ & did not play & player won  \\
\hline
\end{tabular}\end{center}
$$\Pr(\text{dead}|\text{played})$$
$$=\frac{\Pr(\text{dead}\land\text{played})}{\Pr(\text{played})}$$
$$=\frac{5E-7}{5000005E-7} = \frac{1}{1000001}$$


Every child has uniform odds of being a boy or a girl (i.e. 50/50).  What are the odds that Adam has a boy, conditioned on the fact that Adam has exactly two children and at least one of Adam's children is a girl?
\begin{center}\begin{tabular}{ r | c c | }
$\text{ }$ & B $\frac{1}{2}$ & G $\frac{1}{2}$  \\
\hline
B $\frac{1}{2}$ & $\frac{1}{4}$ & $\frac{1}{4}$ \\
$\text{ }$ & BB & GB  \\
$\text{ }$ & $\text{ }$ & $\text{ }$  \\
G $\frac{1}{2}$ & $\frac{1}{4}$ &  $\frac{1}{4}$\\
$\text{ }$ & BG & GG  \\
\hline
\end{tabular}\end{center}
$$\Pr(\text{at least one boy}|\text{at least one girl})$$
$$=\frac{\Pr(\text{at least one boy}\land\text{at least one girl})}{\Pr(\text{at least one girl})}$$
$$=\frac{\frac{1}{2}}{\frac{3}{4}} = \frac{2}{3}$$


Hyper Sleeping Kongo
\begin{center}\begin{tabular}{ r | c c | }
$\text{ }$ & Heads $\frac{1}{2}$ & Tails $\frac{1}{2}$  \\
\hline
$p$ & $\frac{p}{2}$ & $\frac{p}{2}$ \\
$\text{Kongo Says Heads}$ & Win \$1 & Lose \$999,999  \\
$\text{ }$ & $\text{ }$ & $\text{ }$  \\
$1-p$  & $\frac{1-p}{2}$ &  $\frac{1-p}{2}$\\
$\text{Kongo Says Tails}$ & Lose \$1 & Win \$999,999 \\
\hline
\end{tabular}\end{center}
$$\Pr(\text{at least one boy}|\text{at least one girl})$$
$$=\frac{\Pr(\text{at least one boy}\land\text{at least one girl})}{\Pr(\text{at least one girl})}$$
$$=\frac{\frac{1}{2}}{\frac{3}{4}} = \frac{2}{3}$$
}
Let $\Pr(i,j)$ be the mutually exclusive absolute odds of an individual being selected in Round $i$ of a game that rolls snake eyes in Round $j$. Since the game ends the round that we roll snake eyes we have $i\leq j \text{ }\forall \text{ } i\text{,}j \in \mathbf{N}$. If $i > j$ then we were not selected because the game ended before we could play. When $i=j$ then we are devoured by snakes.
The absolute odds of dying is found by adding up all the possible games where we are selected in the final round:
\begin{equation}
    \sum_{j=1}^{\infty} \Pr(j,j) \label{die}
\end{equation} 
The absolute odds of being chosen are the games where  $i\leq j \text{ }\forall \text{ } i\text{,}j \in \mathbf{N}$:
\begin{equation}
   \sum_{j=1}^{\infty} \sum_{i=1}^{j} \Pr(i,j) = \sum_{i=1}^{\infty} \sum_{j=i}^{\infty} \Pr(i,j) \label{chosen} 
\end{equation}

The odds of dying, conditioned upon being selected is simply the ratio of \ref{die} over \ref{chosen}
\begin{equation}
    \frac{\sum_{j=1}^{\infty} \Pr(j,j)}{\sum_{j=1}^{\infty} \sum_{i=1}^{j} \Pr(i,j)}\label{theAnswer}
\end{equation}

NO asserts that the following is the odds of dying, conditioned upon being selected:
$$\sum_{j=1}^{\infty} \frac{\Pr(j,j)p(1-p)^{j-1}}{\sum_{i=1}^{j} \Pr(i,j)} = \frac{1}{1}p + \frac{2}{3}p(1-p) + \frac{4}{7}p(1-p)^2 + ... \approx 0.5218873$$

It is generally accepted that 
$$\frac{\sum_{i=1}^{n} a_i}{\sum_{i=1}^{n} b_i}\nLeftrightarrow \sum_{i=1}^{n} \frac{a_i}{b_i}$$

However we do know that if 
$$\frac{a_i}{b_i}=\frac{a_j}{b_j} \text{   } \forall \text{ } i\text{,}j \in \{1...n\}$$
then
$$\frac{\sum_{i=1}^{n} a_i}{\sum_{i=1}^{n} b_i} = \frac{a_j}{b_j}\text{   } \forall \text{ } i\text{,}j \in \{1...n\}$$

The local odds of dying knowing only that you are selected in Round $i$ and that the game may or may not continue is $p$.  This makes sense since with the locally known information it is just like you are playing a one shot game, observing just one roll of the dice.  Which means
$$\frac{\Pr(i,i)}{\sum_{j=i}^{\infty}\Pr(i,j)} = p \text{  }\forall \text{  } i \in \mathbf{N}$$
which, once accepted, is enough to show that \ref{theAnswer} is also $p$
$$ \frac{\sum_{j=1}^{\infty} \Pr(j,j)}{\sum_{j=1}^{\infty} \sum_{i=1}^{j} \Pr(i,j)} = \frac{\sum_{i=1}^{\infty}\Pr(i,i)}{\sum_{i=1}^{\infty}\sum_{j=i}^{\infty}\Pr(i,j)} = p $$

So lets algebra out the probability of losing, if you are chosen in Round i.
We said that $\Pr(i,j)$ is the mutually exclusive absolute odds of an individual being selected in Round $i$ of a game that rolls snake eyes in Round $j$.  We might need to say what that looks like algebraically. 
\begin{equation}
    \Pr(i,j) = \frac{2^i}{M}p(1-p)^{j-1}
\end{equation}
M is a placeholder for the population size and $2^i$ is the number of players in Round $i$.  $p(1-p)^{j-1}$ is the probability that the game rolls $j-1$ non-snake-eyes before rolling snake eyes on Round $j$

It is only possible to be chosen to play in Round $i$ if $i\leq j$.  You die when $j=i$ otherwise you live for all $j>i$. 
\begin{align*}
 &= \frac{\Pr(i,i)}{\sum_{j=i}^\infty \Pr(i,j)}\\
 &=\frac{\frac{2^{i}p(1-p)^{i-1}}{M}}{\sum_{n=i}^{\infty} \frac{2^{i}p(1-p)^{n-1}}{M}}\\
 &=\frac{p(1-p)^{i-1}}{\sum_{n=i}^{\infty}p(1-p)^{n-1}}\\
 &=\frac{p(1-p)^{i-1}}{p\frac{(1-p)^{i-1}}{1-(1-p)}}\\
 &=\frac{p(1-p)^{i-1}}{(1-p)^{i-1}}\\
 &=p
\end{align*}
Ergo the probability of dying, conditional on being chosen is $p$.
\mycomment{
This should makes sense to us within the narrative that, if you are selected to play in any specific round then the odds of you losing that round is determined by a fair dice roll made for that round and your odds of losing are $p=\frac{1}{36}$. No matter which round you are selected to play in, if you play you will only observe one roll of the dice and, since the dice rolls are independent, the process and your observation are memory-less.
}

RESPONSE: 
        In the St. Petersburg Paradox we have a game with an EV of
        \begin{align*}
            \text{EV}(\text{winnings})&=\frac{1}{2}2 + \frac{1}{4}4 + \frac{1}{8}8 + \frac{1}{16}16 + ...\\
            &=1 + 1 + 1 + 1 + ...\\
            &=\infty
        \end{align*}
        So the question is what would you pay for a chance to win $EV(winnings)=\infty$?  Lets say you are willing to pay $b$ which maximizes your EV(net winnings) for the chance to play.
        \begin{align*}
            \text{EV}(\text{net winnings})&=\text{EV}(\text{winnings})-b\\ 
             &=\infty-b \\
             &= \infty\\
        \end{align*}
        So we should be good paying any amount $b$? But what about
        \begin{align*}
             &\text{EV}(\text{net winnings})\\
             &=\text{EV}(\text{winnings}-b)\\ 
            & =\frac{(2-b)}{2} + \frac{(4-b)}{4} + \frac{(8-b)}{8} + \frac{(16-b)}{16} + ...\\
            & =1 + 1 + 1 + ... -b(\frac{1}{2}+\frac{1}{4}+\frac{1}{8}+\frac{1}{16}+...)\\
            & =\infty-b\\
            \text{so}\\
           b & = \text{  ?  }
        \end{align*}
        It doesn't matter what you would pay, the EV(net winnings) = EV(winnings) = $\infty$

        Does a similar scenario arise in the finitary starting population sizes M passing into the infinitary case?  I perceive the St. Petersburg Paradox to arise from comparing unbound infinite sums. Clearly the YES write up's 
\end{document}