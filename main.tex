%%%%%%%%%%%%%%%%%%%%%%%%%%%%%%%%%%%%%%%%%%%%%%%%%%%%%%%%%%%%%%%%%%%%%%%%%%%%%%%%
%%%%%%%%%%%%%%%%%%%%%%%%% TOGGLES, CONSTANTS, SETTINGS %%%%%%%%%%%%%%%%%%%%%%%%%
%%%%%%%%%%%%%%%%%%%%%%%%%%%%%%%%%%%%%%%%%%%%%%%%%%%%%%%%%%%%%%%%%%%%%%%%%%%%%%%%

\newcount\Chatty  % whether to show notes-to-selves in the pdf
\newcount\Stampy  % whether to show a timestamp in the pdf
\Chatty  = 1 % 0 for final copy; 1 for draft
\Stampy  = 1 % 0 for final copy; 1 for draft

%%%%%%%%%%%%%%%%%%%%%%%%%%%%%%%%%%%%%%%%%%%%%%%%%%%%%%%%%%%%%%%%%%%%%%%%%%%%%%%%
%%%%%%%%%%%%%%%%%%%%%%%%% DOCUMENT CLASS AND PACKAGES %%%%%%%%%%%%%%%%%%%%%%%%%%
%%%%%%%%%%%%%%%%%%%%%%%%%%%%%%%%%%%%%%%%%%%%%%%%%%%%%%%%%%%%%%%%%%%%%%%%%%%%%%%%

\documentclass[article,twocolumn]{memoir}
%\usepackage[utf8]{inputenc} % probably needed if this file has non-ascii chars?
\usepackage{amsmath}
\usepackage{amssymb}
\usepackage{amsthm}
%\usepackage{lipsum} % for dummy text
\usepackage[showseconds=false, showzone=false]{datetime2} % for draft timestamps
\usepackage{xcolor} % used in chatty macros
\usepackage{tikz}

%%%%%%%%%%%%%%%%%%%%%%%%%%%%%%%%%%%%%%%%%%%%%%%%%%%%%%%%%%%%%%%%%%%%%%%%%%%%%%%%
%%%%%%%%%%%%%%%%%%%%%%%%%%%%% COMMANDS AND MACROS %%%%%%%%%%%%%%%%%%%%%%%%%%%%%%
%%%%%%%%%%%%%%%%%%%%%%%%%%%%%%%%%%%%%%%%%%%%%%%%%%%%%%%%%%%%%%%%%%%%%%%%%%%%%%%%

\newcommand{\dreev} [1]{\ifnum\Chatty=1 \textcolor{purple}{dreev:  [#1]} \fi}
% Possible colors: red, blue, purple, orange, teal, etc

\newcommand{\tstamp}{\textcolor{red}{DRAFT~\DTMnow}}

%%%%%%%%%%%%%%%%%%%%%%%%%%%%%%%%%%%%%%%%%%%%%%%%%%%%%%%%%%%%%%%%%%%%%%%%%%%%%%%%
%%%%%%%%%%%%%%%%%%%%%% TITLE, AUTHORS, ABSTRACT, KEYWORDS %%%%%%%%%%%%%%%%%%%%%%
%%%%%%%%%%%%%%%%%%%%%%%%%%%%%%%%%%%%%%%%%%%%%%%%%%%%%%%%%%%%%%%%%%%%%%%%%%%%%%%%

\newcommand{\longtitle}{The Snake Eyes Paradox}
%\newcommand{\shorttitle}{Title Of This Paper}

\title{\HUGE\textbf{\longtitle}}
\author{Daniel M. Reeves\\dreev@beeminder.com}
\date{Last updated: 2023-05-16}

% TODO: how to add keywords?
%\keywords{\LaTeX, Templates, Dummy text}

% TODO: how to add a short version of the title for inclusion in page headers?

%%%%%%%%%%%%%%%%%%%%%%%%%%%%%%%%%%%%%%%%%%%%%%%%%%%%%%%%%%%%%%%%%%%%%%%%%%%%%%%%
%%%%%%%%%%%%%%%%% START DOCUMENT, SET UP HEADERS, DO MAKETITLE %%%%%%%%%%%%%%%%%
%%%%%%%%%%%%%%%%%%%%%%%%%%%%%%%%%%%%%%%%%%%%%%%%%%%%%%%%%%%%%%%%%%%%%%%%%%%%%%%%

\begin{document}
\pagestyle{headings}
\maketitle

%%%%%%%%%%%%%%%%%%%%%%%%%%%%%%%%%%%%%%%%%%%%%%%%%%%%%%%%%%%%%%%%%%%%%%%%%%%%%%%%
%%%%%%%%%%%%%%%%%%%%%%%%%%%%%%%% MAIN DOCUMENT %%%%%%%%%%%%%%%%%%%%%%%%%%%%%%%%%
%%%%%%%%%%%%%%%%%%%%%%%%%%%%%%%%%%%%%%%%%%%%%%%%%%%%%%%%%%%%%%%%%%%%%%%%%%%%%%%%

%\begin{abstract}
%The answer is 1/36.
%\end{abstract}

%%%%%%%%%%%%%%%%%%%%%%%%%%%%%%%%%%%%%%%%%%%%%%%%%%%%%%%%%%%%%%%%%%%%%%%%%%%%%%%%

\chapter*{Problem Statement}

You are offered a gamble where a pair of six-sided dice are rolled and unless they come up snake eyes you get a bajillion dollars. 
If they do come up snake eyes, you're devoured by snakes.

So far it sounds like you have a 1/36 chance of dying, right?

Now the twist. 
First, I gather up an unlimited number of people willing to play the game. 
I take 1 person from that pool and let them play. 
Then I take 2 people and have them play together, where they share a dice roll and either get the bajillion dollars each or both get devoured. 
Then I do the same with 4 people, and then 8, 16, and so on.

At some point one of those groups will be devoured by snakes and then I stop.

Is the probability that you'll die, given that you're chosen to play, still 1/36?

Argument for NO: 
Due to the doubling, the final group of people that die is slightly bigger than all the surviving groups put together. 
So if you're chosen to play you have about a 50\% chance of dying!

Argument for YES: 
The dice rolls are independent and whenever you're chosen, whatever happened in earlier rounds is irrelevant.
Your chances of death are the chances of snake eyes on your round: 1/36.

So which is it? 
What's your probability of dying, conditional on being chosen to play?

\newpage

Some clarifications:

\begin{enumerate}
\item The game is not adversarial and the dice rolls are independent and truly random.
\item Choosing each group also happens uniformly x randomly and without replacement.
\item The question is about the unrealistic case of an unbounded number of people but we can cap it and say that if no one has died after N rounds then the game ends and no one dies. 
We just need to then find the limit as N goes to infinity, in which case the probability that no one dies goes to zero.
\item Importantly, in the finite version it's possible for no one to die. 
But the probability of that approaches zero as the size of the pool approaches infinity.
\end{enumerate}

\emph{Don't turn the page until you're ready for the solution!}

\newpage

\chapter*{Solution}

We want the probability that you die given that you're chosen to play, 
$\Pr(\text{death} \mid \text{chosen})$.
It's true that roughly half the people who are chosen die, but let's Bayes it out carefully:

\begin{equation}
\begin{split}
\Pr(\text{death} \mid \text{chosen}) & =
\frac{\Pr(\text{chosen} \mid \text{death}) \Pr(\text{death})}{\Pr(\text{chosen})} \\
& = \frac{1\cdot\Pr(\text{death})}{\Pr(\text{chosen})}.
\end{split}
\end{equation}

In the uncapped case, that conditional probability is undefined. 
If there's no cap then you're part of an infinite pool. 
So you have a 0\% chance of being chosen and a 0\% chance of dying. 
The probability we want is 0/0. 
\emph{*robot-with-smoke-coming-out-of-its-ears-emoji*}

Since we can't directly calculate the probability in the infinite case, we have to take a limit.

To get a feel for it, suppose I'm 1 person in a huge but finite pool. 
Now suppose I'm actually chosen. 
There are two ways that can happen: 
the pool runs out and everyone survives, or 
the pool doesn't run out and I have about a 50\% chance of dying. 
But knowing that I'm chosen is Bayesian evidence that we had many, many rounds of survival. 
If an early group died then most of the pool wasn't chosen, so probably I wasn't chosen.

Or, using the old think-like-a-Bayesian trick where you shift your probability in light of evidence by seeing how surprised you'd be in various universes by that evidence:

If an early group died then most people aren't chosen and in that universe I'm surprised to be chosen. 
If *no* group died then everyone was chosen and in that universe I'm fully unsurprised that I was chosen. 
That's the sense in which being chosen is Bayesian evidence that more people survived. 
In particular it's at least weak evidence that everyone survived.

So even with an absurdly huge pool of people, where there's \emph{essentially} a zero percent chance of everyone surviving, if I know I was chosen---which itself has near zero probability---then that means we're more likely in that essentially-zero-percent probability universe where everyone survives.

So now it's just a matter of Bayes-ing it out to see what Pr(death | chosen) is exactly, in the version where we stop after N rounds.
Once we have that, we can take the limit as n goes to infinity.

So here it is.

Let $M$ be the size of the pool:

$$M = \sum_{i=0}^{N-1} 2^i = 2^N-1$$

And let $p$ be the probability of snake eyes, 1/36.
We can now compute the probability of being chosen by summing up 
(1) the probability you're chosen for the first round, $1/M$ plus 
(2) the probability that the first group survives, $1-p$, and that you're chosen for the 2nd round, $2/M$ plus 
(3) the probability that the first two groups survive and you're chosen for the 3rd round, etc.
Writing that out as an equation gives this:

\begin{align*}
\Pr(\text{chosen}) & = 
\begin{aligned}[t]
\tfrac{1}{M} & + (1-p)\tfrac{2}{M} \\
& + (1-p)^2\tfrac{4}{M} \\
& + (1-p)^3\tfrac{8}{M} \\
& + \ldots \\
& + (1-p)^{N-1}\frac{2^{N-1}}{M}
\end{aligned} \\
& = \sum_{i=0}^{N-1} \tfrac{1}{M} 2^i(1-p)^i.
\end{align*}

\begin{equation}
\begin{split}
\Pr(\text{chosen}) & = \tfrac{1}{M} + (1-p)\tfrac{2}{M} \\
& \phantom{mmii} + (1-p)^2\tfrac{4}{M} \\
& \phantom{mmii} + (1-p)^3\tfrac{8}{M} \\
& \phantom{mmii} + ... \\
& \phantom{mmii} + (1-p)^{N-1}\frac{2^{N-1}}{M} \\
& = \sum_{i=0}^{N-1} \tfrac{1}{M} 2^i(1-p)^i.
\end{split}
\end{equation}

For $\Pr(\text{death})$ the calculation is very similar but every term is multiplied by $p$.
To die, you have to be chosen and then roll snake eyes.
This can happen on any round, all of which are mutually exclusive.
We can then factor that $p$ out and we have

$$
\Pr(\text{death}) = p\cdot\Pr(\text{chosen}).
$$

Which means

\begin{equation}
\begin{split}
\Pr(\text{death} \mid \text{chosen}) & = 
\frac{\Pr(\text{death})}{\Pr(\text{chosen})} \\
& = \frac{p\cdot\Pr(\text{chosen})}{\Pr(\text{chosen})} = 
p.
\end{split}
\end{equation}

It doesn't depend on $N$ at all!
So the limit as $N$ goes to infinity is just... 
$p$ or 1/36, the probability of rolling snake eyes.
\qedsymbol{}

%%%%%%%%%%%%%%%%%%%%%%%%%%%%%%%%%%%%%%%%%%%%%%%%%%%%%%%%%%%%%%%%%%%%%%%%%%%%%%%%

\chapter*{Acknowledgments}

Thanks to 
Greta Goodwin, 
Bethany Soule, 
Christopher Moravec,
and Stan Wagon, as well as
CesiumLifeJacket and Nix and others in the PEAR Discord for helpful discussion.

Thanks also to Manifold Markets.

\end{document}

%%%%%%%%%%%%%%%%%%%%%%%%%%%%%%%%%%%%%%%%%%%%%%%%%%%%%%%%%%%%%%%%%%%%%%%%%%%%%%%%
%%%%%%%%%%%%%%%%%%%%%%%%%%%%%%%% SCRATCH NOTES %%%%%%%%%%%%%%%%%%%%%%%%%%%%%%%%%
%%%%%%%%%%%%%%%%%%%%%%%%%%%%%%%%%%%%%%%%%%%%%%%%%%%%%%%%%%%%%%%%%%%%%%%%%%%%%%%%

To review, the uncapped version of the problem is not well posed because we have to condition on something (being chosen to play out of an infinite pool) that has probability zero.
Relatedly, the expected value of the number of people chosen to play is infinite, as is the expected number of deaths.
So we need to consider a finite version and take the limit.
But in the finite version there are two ways you can survive.
First, if the game ends in death and you're chosen to play then you have about a 50% chance.
Second, if the game ends by exhausting the pool, you definitely are chosen and definitely survive.
The probability of exhausting the pool is $(1-p)^N$ -- not rolling snake eyes $N$ times in a row.


https://link.springer.com/article/10.1023/A:1005100407551
